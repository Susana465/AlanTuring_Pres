\documentclass[t]{beamer}

\usetheme{TuringLight}
%\usetheme{TuringDark}

% Presentation data
\subtitle{Ethical standards and reproducibility of computer models in Neurobiology}
\date{19/01/2022}
\author{Susana Roman Garcia}

% Uncomment any of these lines below to set custom size for each of the font sizes.
% The default value is shown in the comment.
%\setlength{\titlefontsize}{6.875\basefontsize}
%\setlength{\subtitlefontsize}{4.375\basefontsize}
%\setlength{\frametitlesize}{2.625\basefontsize}
%\setlength{\framesubtitlesize}{1.625\basefontsize}
%\setlength{\bodytextsize}{2\basefontsize}
%\setlength{\blocktitlesize}{\bodytextsize}
%\setlength{\blockbodysize}{\bodytextsize}

% Start document
\begin{document}

% Title slide (details filled from presentation data fields above)
\begin{frame}
	\titlepage
\end{frame}

% Imprint slide (e.g. about the institute / opening quote)
\begin{frame}{16 week (Sep 22 - Jan 23) project in collaboration with:}
    \hfill \break
	\includegraphics[height=0.25\paperheight,keepaspectratio]{OLS_logo.png}
        \includegraphics[height=0.25\paperheight,keepaspectratio]{uoe_logo.png}
        \includegraphics[height=0.25\paperheight,keepaspectratio]{ATI_logo.png}
\end{frame}

\begin{frame}{With special thanks to:}
    \begin{itemize}
        \item Siobhan Mackenzie Hall, OLS mentor.

        \item Melanie Stefan, David Sterratt, Nicola Romano, PhD supervisors.

        \item Claudia Fischer, Alan Turing Institute guidance.

        \item Everyone at OLS and anyone who came along the journey.
    \end{itemize}
\end{frame}

\begin{frame}{Contents}
	\tableofcontents
\end{frame}

% Section divider slide
\section{1.Project background}
\subsection{1.1. Aim and motivation of this project:}
\begin{frame}{1.1. Aim and motivation of this project:}
	\begin{block}{Offer a case study example of how to create a PhD that looks at reproducibility and ethics as part of the process, not as an add-on.}
        \hfill \break
  		\begin{itemize}    
  			\item To tackle reproducibility issues and stop wasting money, time and resources in general.
  			\item Reproducibility only makes sense if bias is accounted for too. Otherwise, oppressive biases carry on without being questioned.
  		\end{itemize}    
	\end{block}
\end{frame}

\subsection{1.2. Missions and goals set in the beginning.}
\begin{frame}{1.2. Missions and goals set in the beginning.}
	\begin{block}{Create a written guide for looking at bias and reproducibility in a PhD.}
  		\begin{itemize}    
  			\item There are already guides out there.
  			\item Make my PhD a case study, an example of how to implement these thoughts. 
  		\end{itemize}    
	\end{block}
 
	\begin{block}{Look at bias and reproducibility in Computational Neurobiology.}
  		\begin{itemize}    
  			\item Quantify bias in literature to showcase importance.
  			\item Lots of different types of bias, which one to choose? 
                \item Speciesism.
  		\end{itemize}    
	\end{block}
        \begin{block}{Publish results.}   
	\end{block}
\end{frame}

\section{2. Process}
\subsection{2.1. Goals achieved, key understandings.}
\begin{frame}{2.1. Goals achieved, key understandings.}
	\begin{block}{Create a written guide for looking at bias and reproducibility in a PhD.}
  		\begin{itemize}    
  			\item This takes a lot more time than I had anticipated.
  			\item Will embed with PhD thesis, in process.
  		\end{itemize}    
	\end{block}
 
	\begin{block}{Look at bias and reproducibility in Computational Neurobiology.}
  		\begin{itemize}    
  			\item Quantifying bias in literature to showcase importance.
  			\item Lots of brain storming. Lots of hours spent deciding which questions to ask.
                \item Lots of time spent looking for templates of other people's work.
  		\end{itemize}    
	\end{block}
        \begin{block}{Publish results...contained in GitHub for now.}   
	\end{block}
\end{frame}

\subsection{2.2. Was this the same as goals originally set?}
\begin{frame}{2.2. Was this the same as goals originally set?}
	\begin{block}{Kind of...}
  		\begin{itemize}    
  			\item I ended up learning a lot more about how GitHub works,
  			\item About licencing my work,
                \item About making more open, accessible work,
                \item Making contacts, reading a lot.
  		\end{itemize}    
	\end{block}
\end{frame}

\section{3. Outcomes}
\subsection{3.1. Project Collaboration proposal at the Turing Institute.}
% Skeleton double-column text slide (two text columns)
\begin{frame}{3.1. Project Collaboration proposal at the Turing Institute.}
	\begin{columns}[T,totalwidth=\textwidth]
  		\begin{column}{0.45\textwidth}
  			\begin{block}{Block 1 title}
    				Body copy
    				\begin{itemize}    
    					\item Level 1 bullet
  					\begin{itemize}
  						\item Level 2 bullet
  						\begin{itemize}
  							\item Level 3 bullet
  						\end{itemize}
  					\end{itemize}
    					\item Item Level 1 bullet 2
    				\end{itemize}  
			\end{block}
  		\end{column} %
  		\begin{column}{0.45\textwidth}
  			\begin{block}{Block 2 title}
    				Body copy
    				\begin{itemize}    
    					\item Level 1 bullet
  					\begin{itemize}
  						\item Level 2 bullet
  						\begin{itemize}
  							\item Level 3 bullet
  						\end{itemize}
  					\end{itemize}
    					\item Item Level 1 bullet 2
    				\end{itemize}  
			\end{block}
  		\end{column}%
	\end{columns}
\end{frame}

% Skeleton double-column text and image slide (left-hand text column, right-hand image)
\begin{frame}{Double column text and image page title}
	\begin{columns}[T,totalwidth=\textwidth]
  		\begin{column}{0.45\textwidth}
  			\begin{block}{Block 1 title}
    				Body copy
    				\begin{itemize}    
    					\item Level 1 bullet
  					\begin{itemize}
  						\item Level 2 bullet
  						\begin{itemize}
  							\item Level 3 bullet
  						\end{itemize}
  					\end{itemize}
    					\item Item Level 1 bullet 2
    				\end{itemize}  
			\end{block}
  		\end{column} %
  		\begin{column}{0.45\textwidth}
			\begin{figure}
				\vspace{-\blocktitlesize}
				\includegraphics[height=0.65\paperheight,keepaspectratio]{drawing-on-glass.jpg}
			\end{figure}
  		\end{column}%
	\end{columns}
\end{frame}

% End slide 
\begin{frame}	
	\finalpage
\end{frame}

\end{document}
